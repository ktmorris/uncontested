% Options for packages loaded elsewhere
\PassOptionsToPackage{unicode}{hyperref}
\PassOptionsToPackage{hyphens}{url}
%
\documentclass[
  ignorenonframetext,
]{beamer}
\usepackage{pgfpages}
\setbeamertemplate{caption}[numbered]
\setbeamertemplate{caption label separator}{: }
\setbeamercolor{caption name}{fg=normal text.fg}
\beamertemplatenavigationsymbolsempty
% Prevent slide breaks in the middle of a paragraph
\widowpenalties 1 10000
\raggedbottom
\setbeamertemplate{part page}{
  \centering
  \begin{beamercolorbox}[sep=16pt,center]{part title}
    \usebeamerfont{part title}\insertpart\par
  \end{beamercolorbox}
}
\setbeamertemplate{section page}{
  \centering
  \begin{beamercolorbox}[sep=12pt,center]{part title}
    \usebeamerfont{section title}\insertsection\par
  \end{beamercolorbox}
}
\setbeamertemplate{subsection page}{
  \centering
  \begin{beamercolorbox}[sep=8pt,center]{part title}
    \usebeamerfont{subsection title}\insertsubsection\par
  \end{beamercolorbox}
}
\AtBeginPart{
  \frame{\partpage}
}
\AtBeginSection{
  \ifbibliography
  \else
    \frame{\sectionpage}
  \fi
}
\AtBeginSubsection{
  \frame{\subsectionpage}
}
\usepackage{lmodern}
\usepackage{amssymb,amsmath}
\usepackage{ifxetex,ifluatex}
\ifnum 0\ifxetex 1\fi\ifluatex 1\fi=0 % if pdftex
  \usepackage[T1]{fontenc}
  \usepackage[utf8]{inputenc}
  \usepackage{textcomp} % provide euro and other symbols
\else % if luatex or xetex
  \usepackage{unicode-math}
  \defaultfontfeatures{Scale=MatchLowercase}
  \defaultfontfeatures[\rmfamily]{Ligatures=TeX,Scale=1}
\fi
\usetheme[]{Berlin}
% Use upquote if available, for straight quotes in verbatim environments
\IfFileExists{upquote.sty}{\usepackage{upquote}}{}
\IfFileExists{microtype.sty}{% use microtype if available
  \usepackage[]{microtype}
  \UseMicrotypeSet[protrusion]{basicmath} % disable protrusion for tt fonts
}{}
\makeatletter
\@ifundefined{KOMAClassName}{% if non-KOMA class
  \IfFileExists{parskip.sty}{%
    \usepackage{parskip}
  }{% else
    \setlength{\parindent}{0pt}
    \setlength{\parskip}{6pt plus 2pt minus 1pt}}
}{% if KOMA class
  \KOMAoptions{parskip=half}}
\makeatother
\usepackage{xcolor}
\IfFileExists{xurl.sty}{\usepackage{xurl}}{} % add URL line breaks if available
\IfFileExists{bookmark.sty}{\usepackage{bookmark}}{\usepackage{hyperref}}
\hypersetup{
  pdftitle={The Up-Ballot Implications of Uncontested US House Races},
  pdfauthor={Kevin Morris; Peter Miller},
  hidelinks,
  pdfcreator={LaTeX via pandoc}}
\urlstyle{same} % disable monospaced font for URLs
\newif\ifbibliography
\setlength{\emergencystretch}{3em} % prevent overfull lines
\providecommand{\tightlist}{%
  \setlength{\itemsep}{0pt}\setlength{\parskip}{0pt}}
\setcounter{secnumdepth}{-\maxdimen} % remove section numbering
\newlength{\cslhangindent}
\setlength{\cslhangindent}{1.5em}
\newenvironment{cslreferences}%
  {\setlength{\parindent}{0pt}%
  \everypar{\setlength{\hangindent}{\cslhangindent}}\ignorespaces}%
  {\par}

\title{The Up-Ballot Implications of Uncontested US House Races}
\subtitle{The Case of 2018}
\author{Kevin Morris \and Peter Miller}
\date{Southern Political Science Association, 2020}
\institute{Brennan Center for Justice}

\begin{document}
\frame{\titlepage}

\begin{frame}{Overview}
\protect\hypertarget{overview}{}
\begin{itemize}[<+->]
\tightlist
\item
  Question: Does residing in an uncontested U.S. House district have an
  effect on casting a ballot at all? What are the ``upward" turnout
  effects of an absent down-ballot race?
\end{itemize}

\begin{itemize}[<+->]
\tightlist
\item
  Data: Voter registration data from the six states with multiple
  uncontested House races in the 2018 election

  \begin{itemize}[<+->]
  \tightlist
  \item
    and Wisconsin (for the legislative election)
  \end{itemize}
\end{itemize}

\begin{itemize}[<+->]
\tightlist
\item
  Method: A genetic matching model comparing registrants in contested
  and uncontested districts
\end{itemize}

\begin{itemize}[<+->]
\tightlist
\item
  Findings: The turnout effect is negative in five of the seven states
  (not CA and GA)

  \begin{itemize}[<+->]
  \tightlist
  \item
    and larger for the ``represented party" in five of the seven (not
    Texas and Wisconsin)
  \end{itemize}
\end{itemize}
\end{frame}

\begin{frame}{Prior Literature I}
\protect\hypertarget{prior-literature-i}{}
\begin{itemize}[<+->]
\tightlist
\item
  Studies of uncontested House races

  \begin{itemize}[<+->]
  \tightlist
  \item
    More common in the South and predicted by incumbent's vote share in
    prior elections (Squire \protect\hyperlink{ref-Squire1989}{1989};
    Wrighton and Squire \protect\hyperlink{ref-Wrighton1997}{1997})
  \end{itemize}
\end{itemize}

\begin{itemize}[<+->]
\tightlist
\item
  Patterns in American voting behavior

  \begin{itemize}[<+->]
  \tightlist
  \item
    Surge and decline in presidential and midterm election years
    (A. Campbell \protect\hyperlink{ref-Campbell1960}{1960}; J. Campbell
    \protect\hyperlink{ref-Campbell1991}{1991})
  \item
    Roll-off in congressional and ballot measure contests (Wattenberg,
    McAllister, and Salvanto
    \protect\hyperlink{ref-Wattenberg2000}{2000}; Bullock and Dunn
    \protect\hyperlink{ref-Bullock1996}{1996}; Hall and Aspin
    \protect\hyperlink{ref-Hall1987}{1987})
  \item
    Party-based mobilization (Wielhouwer and Lockerbie
    \protect\hyperlink{ref-Wielhouwer1994}{1994})
  \end{itemize}
\end{itemize}
\end{frame}

\begin{frame}{Prior Literature II}
\protect\hypertarget{prior-literature-ii}{}
\begin{itemize}[<+->]
\tightlist
\item
  Redistricting and representation

  \begin{itemize}[<+->]
  \tightlist
  \item
    Sorting (Bishop \protect\hyperlink{ref-Bishop2009}{2009}) and ethnic
    turnout in majority-minority districts (Griffin and Keane
    \protect\hyperlink{ref-Griffin2006}{2006}; Fairdosi and Rogowski
    \protect\hyperlink{ref-Fairdosi2015}{2015})
  \item
    ``packing'' and ``cracking'' for advantage
  \item
    Redistricting reduces turnout by disrupting incumbent recall
    (Winburn and Wagner \protect\hyperlink{ref-Winburn2009}{2009}; Hayes
    and McKee \protect\hyperlink{ref-Hayes2009}{2009})
  \end{itemize}
\end{itemize}
\end{frame}

\begin{frame}{Data}
\protect\hypertarget{data}{}
\begin{itemize}[<+->]
\tightlist
\item
  Registered Voter Files

  \begin{itemize}[<+->]
  \tightlist
  \item
    Some directly from the state, some from Aristotle and L2
  \item
    These include a host of information about voters' age, sex, address,
    (sometimes) race, and others
  \end{itemize}
\end{itemize}

\begin{itemize}[<+->]
\tightlist
\item
  SmartyStreets Geocoder + Census Data
\end{itemize}

\begin{itemize}[<+->]
\tightlist
\item
  We leverage the Bayesian racial probability estimate methodology
  developed by Imai and Khanna (\protect\hyperlink{ref-Imai2016}{2016})
  for states without self-reported race data.
\end{itemize}
\end{frame}

\begin{frame}{Methodology}
\protect\hypertarget{methodology}{}
\begin{itemize}[<+->]
\tightlist
\item
  Genetic matching developed by Sekhon
  (\protect\hyperlink{ref-Sekhon2011}{2011}). The weights are estimated
  using a random 1 percent sample of treated and untreated observations.
\end{itemize}

\begin{itemize}[<+->]
\tightlist
\item
  Each treated voter is matched to 25 untreated voter, and matching is
  done with replacement.
\end{itemize}

\begin{itemize}[<+->]
\tightlist
\item
  The matching procedure resulted in strong improvements across all the
  factors on which we match.
\end{itemize}
\end{frame}

\begin{frame}{Match Output}
\protect\hypertarget{match-output}{}
\begin{center}\includegraphics[width=0.49\linewidth,height=0.49\textheight]{../temp/perc_white_spsa} \includegraphics[width=0.49\linewidth,height=0.49\textheight]{../temp/perc_black_spsa} \end{center}
\end{frame}

\begin{frame}{Match Output}
\protect\hypertarget{match-output-1}{}
\begin{center}\includegraphics[width=0.49\linewidth,height=0.49\textheight]{../temp/income_spsa} \includegraphics[width=0.49\linewidth,height=0.49\textheight]{../temp/perc_dems_spsa} \end{center}
\end{frame}

\begin{frame}{Regression Results}
\protect\hypertarget{regression-results}{}
\begin{center}\includegraphics[width=0.85\linewidth,height=0.85\textheight]{../output/plot1} \end{center}
\end{frame}

\begin{frame}{Regression Results}
\protect\hypertarget{regression-results-1}{}
\begin{center}\includegraphics[width=0.85\linewidth,height=0.85\textheight]{../output/plot2} \end{center}
\end{frame}

\begin{frame}{Regression Results}
\protect\hypertarget{regression-results-2}{}
\begin{center}\includegraphics[width=0.85\linewidth,height=0.85\textheight]{../output/plot3} \end{center}
\end{frame}

\begin{frame}{Regression Results}
\protect\hypertarget{regression-results-3}{}
\begin{center}\includegraphics[width=0.85\linewidth,height=0.85\textheight]{../output/plot4} \end{center}
\end{frame}

\begin{frame}{Conclusions and Next Steps}
\protect\hypertarget{conclusions-and-next-steps}{}
\begin{itemize}[<+->]
\tightlist
\item
  Explore what might be at play in California and Georgia

  \begin{itemize}[<+->]
  \tightlist
  \item
    The CCES 2018 data might be able to show what is going on if
    incumbent recall is the mechanism (but not efficacy)
  \item
    Or maybe the top-two primary in California might explain some of the
    difference
  \end{itemize}
\end{itemize}

\begin{itemize}[<+->]
\tightlist
\item
  The represented party tends to be more likely to vote than the
  unrepresented party

  \begin{itemize}[<+->]
  \tightlist
  \item
    Except in Texas and Wisconsin, which both had competitive statewide
    races
  \end{itemize}
\end{itemize}

\begin{itemize}[<+->]
\tightlist
\item
  What are the overtime effects?

  \begin{itemize}[<+->]
  \tightlist
  \item
    Replicate the same methods in 2020 and 2022
  \end{itemize}
\end{itemize}
\end{frame}

\begin{frame}[allowframebreaks,allowframebreaks]{References}
\protect\hypertarget{references}{}
\hypertarget{refs}{}
\begin{cslreferences}
\leavevmode\hypertarget{ref-Bishop2009}{}%
Bishop, Bill. 2009. \emph{The Big Sort: Why the Clustering of
Like-Minded America Is Tearing Us Apart}. Mariner Books.

\leavevmode\hypertarget{ref-Bullock1996}{}%
Bullock, Charles, and Richard Dunn. 1996. ``Election Roll-Off: A Test of
Three Explanations.'' \emph{Urban Affairs Review} 32: 71--86.

\leavevmode\hypertarget{ref-Campbell1960}{}%
Campbell, Angus. 1960. ``Surge and Decline: A Study of Electoral
Change.'' \emph{Public Opinion Quarterly} 24: 397--418.

\leavevmode\hypertarget{ref-Campbell1991}{}%
Campbell, James. 1991. ``The Presidential Surge and Its Midterm Decline
in Congressional Elections, 1868-1988.'' \emph{Journal of Politics} 53:
477--87.

\leavevmode\hypertarget{ref-Fairdosi2015}{}%
Fairdosi, Amir Shawn, and Jon Rogowski. 2015. ``Candidate Race,
Partisanship, and Political Participation: When Do Black Candidates
Increase Black Turnout?'' \emph{Political Research Quarterly} 68:
337--49.

\leavevmode\hypertarget{ref-Griffin2006}{}%
Griffin, John, and Michael Keane. 2006. ``Descriptive Representation and
the Composition of African American Turnout.'' \emph{American Journal of
Political Science} 50: 998--1012.

\leavevmode\hypertarget{ref-Hall1987}{}%
Hall, William, and Larry Aspin. 1987. ``The Roll-Off Effect in Judicial
Retention Elections.'' \emph{Social Science Journal} 24: 415--27.

\leavevmode\hypertarget{ref-Hayes2009}{}%
Hayes, Danny, and Seth McKee. 2009. ``The Participatory Effects of
Redistricting.'' \emph{American Journal of Political Science} 53:
1006--23.

\leavevmode\hypertarget{ref-Imai2016}{}%
Imai, Kosuke, and Kabir Khanna. 2016. ``Improving Ecological Inference
by Predicting Individual Ethnicity from Voter Registration Records.''
\emph{Political Analysis} 24 (2): 263--72.
\url{https://doi.org/10.1093/pan/mpw001}.

\leavevmode\hypertarget{ref-Sekhon2011}{}%
Sekhon, Jasjeet S. 2011. ``Multivariate and Propensity Score Matching
Software with Automated Balance Optimization: TheMatchingPackage forR.''
\emph{Journal of Statistical Software} 42 (7).
\url{https://doi.org/10.18637/jss.v042.i07}.

\leavevmode\hypertarget{ref-Squire1989}{}%
Squire, Peverill. 1989. ``Competition and Uncontested Seats in U.s.
House Elections.'' \emph{Legislative Studies Quarterly} 14: 281--95.

\leavevmode\hypertarget{ref-Wattenberg2000}{}%
Wattenberg, Martin, Ian McAllister, and Anthony Salvanto. 2000. ``How
Voting Is Like Taking an Sat Test: An Analysis of American Voter
Rolloff.'' \emph{American Politics Research} 28: 234--50.

\leavevmode\hypertarget{ref-Wielhouwer1994}{}%
Wielhouwer, Peter, and Brad Lockerbie. 1994. ``Party Contacting and
Political Participation, 1952-90.'' \emph{American Journal of Political
Science} 38: 211--29.

\leavevmode\hypertarget{ref-Winburn2009}{}%
Winburn, Jonathan, and Michael Wagner. 2009. ``Carving Voters Out:
Redistricting's Influence on Political Information, Turnout, and Voting
Behavior.'' \emph{Political Research Quarterly} 63: 373--86.

\leavevmode\hypertarget{ref-Wrighton1997}{}%
Wrighton, Mark, J, and Peverill Squire. 1997. ``Uncontested Seats and
Electoral Competition for the U.s. House of Representatives over Time.''
\emph{Journal of Politics} 59: 452--68.
\end{cslreferences}
\end{frame}

\end{document}
